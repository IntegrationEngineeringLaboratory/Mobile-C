\rhead{\bf MC\_GetAgentArrivalTime()}
\noindent
\vspace{5pt}
\rule{6.5in}{0.015in}
\noindent
\phantomsection
{\LARGE \bf MC\_GetAgentArrivalTime()\index{MC\_GetAgentArrivalTime()}}\\
\addcontentsline{toc}{section}{MC\_GetAgentArrivalTime()}
\label{api:MC_GetAgentArrivalTime()}

\noindent
{\bf Synopsis}\\
{\bf \#include $<$libmc.h$>$}\\
{\bf time\_t MC\_GetAgentArrivalTime}({\bf MCAgent\_t} $agent$);\\

\noindent
{\bf Purpose}\\
Get the agent's arrival time.\\

\noindent
{\bf Return Value}\\
This function returns a valid \texttt{time\_t} type variable under unix,
or a valid \texttt{SYSTEMTIME} type variable under Microsoft Windows.

\noindent
{\bf Parameters}
\vspace{-0.1in}
\begin{description}
\item
\begin{tabular}{p{10 mm}p{145 mm}} 
$agent$ & An initialized mobile agent.
\end{tabular}
\end{description}

\noindent
{\bf Description}\\
Each agent that arrives at an agency keeps a record of the system time
at the point at which is arrives. This API function is used to access
that data. \\

\noindent
{\bf Example}\\
\noindent
% FIXME: Need an example here
%{\footnotesize\verbatiminput{../demos/multiple_agency_example/server.c}}

\noindent
{\bf See Also}\\

%\CPlot::\DataThreeD(), \CPlot::\DataFile(), \CPlot::\Plotting(), \plotxy().\\
