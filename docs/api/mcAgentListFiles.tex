\rhead{\bf mc\_AgentListFiles()}
\noindent
\vspace{5pt}
\rule{6.5in}{0.015in}
\noindent
\phantomsection
{\LARGE \bf mc\_AgentListFiles()\index{mc\_AgentListFiles()}}\\
\addcontentsline{toc}{section}{mc\_AgentListFiles()}

\noindent
{\bf Synopsis}\\
{\bf \#include $<$libmc.h$>$}\\
{\bf int mc\_AgentListFiles}({\bf MCAgent\_t} $agent$, 
                                  {\bf int} $tasknum$,
                                  {\bf char***} $names$ \texttt{/* OUT */},
                                  {\bf int*} $numfiles$ \texttt{/* OUT */}
																	);\\

\noindent
{\bf Purpose}\\
This funciton is used to list the files attached to an agent's task.\\

\noindent
{\bf Return Value}\\
The function returns 0 on success or a non-zero error code on failure.\\

\noindent
{\bf Parameters}
\vspace{-0.1in}
\begin{description}
\item
\begin{tabular}{p{30 mm}p{125 mm}} 
$agent$ & A fully initialized agent handle.\\
$tasknum$ & The selected task to list attached files.\\
$names$ & A two dimensional array to fill with names of attached files. This
data structure will need to be freed by the user after usage.\\
$numfiles$ & An integer to fill with the number of files attached to the task.
\end{tabular}
\end{description}

\noindent
{\bf Description}\\
This function is used to retrieve the names of files that are attached to an
agent's task. The names may be used in other API function called such as
\texttt{mc\_AgentRetrieveFile()} or \texttt{mc\_AgentRetrieveFile()}.

\noindent
{\bf Example}\\
\noindent
Please see the example listed with the documentation for
\texttt{mc\_AgentAttachFile()}.

\noindent
{\bf See Also}\\
\texttt{mc\_AgentRetrieveFile(), mc\_AgentAttachFiles()}

%\CPlot::\DataThreeD(), \CPlot::\DataFile(), \CPlot::\Plotting(), \plotxy().\\
