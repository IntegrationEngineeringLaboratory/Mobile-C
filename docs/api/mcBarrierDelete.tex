\rhead{\bf mc\_BarrierDelete()}
\noindent
\vspace{5pt}
\rule{6.5in}{0.015in}
\noindent
\phantomsection
{\LARGE \bf mc\_BarrierDelete()\index{mc\_BarrierDelete()}}\\
\addcontentsline{toc}{section}{mc\_BarrierDelete()}
\label{api:mc_BarrierDelete()}

\noindent
{\bf Synopsis}\\
{\bf int mc\_BarrierDelete}({\bf int} $id$);\\

\noindent
{\bf Purpose}\\
This function deletes a previously initialized Mobile-C Barrier variable.
 \\

\noindent
{\bf Return Value}\\
This function returns 0 on success, or non-zero if the id could not be found. \\

\noindent
{\bf Parameters}
\vspace{-0.1pt}
\begin{description}
\item
\begin{tabular}{p{10 mm}p{145 mm}} 
$id$ & The id of the barrier to delete. 
\end{tabular}
\end{description}

\noindent
{\bf Description}\\
This function deletes a previously initialized variable. Care should be taken
when calling this function. If there are any agents or threads blocked
by a barrier that is deleted, they may remain blocked forever.
    \\

\noindent
{\bf Example}\\
Please see the example located at the directory
\texttt{ mobilec/demos/mc\_barrier\_example/ }. \\

\noindent

\noindent
{\bf See Also}\\
mc\_Barrier(), mc\_BarrierInit(). \\

%\CPlot::\DataThreeD(), \CPlot::\DataFile(), \CPlot::\Plotting(), \plotxy().\\
