\rhead{\bf MC\_GetAllAgents()}
\noindent
\vspace{5pt}
\rule{6.5in}{0.015in}
\noindent
\phantomsection
{\LARGE \bf MC\_GetAllAgents()\index{MC\_GetAllAgents()}}\\
\addcontentsline{toc}{section}{MC\_GetAllAgents()}
\label{api:MC_GetAllAgents()}

\noindent
{\bf Synopsis}\\
{\bf \#include $<$libmc.h$>$}\\
{\bf int MC\_GetAllAgents}({\bf MCAgency\_t} $agency$, {\bf MCAgent\_t**} $agents$, {\bf int*} $num\_agents$);\\

\noindent
{\bf Purpose}\\
Retrieve an array of all agents currently registered on an agency. \\

\noindent
{\bf Return Value}\\
The function returns 0 on success and non-zero otherwise.\\

\noindent
{\bf Parameters}
\vspace{-0.1in}
\begin{description}
\item
\begin{tabular}{p{10 mm}p{145 mm}} 
$agency$ & An initialized agency handle to get agents from.\\
$agents$ & The address of a \texttt{MCAgent\_t*} type variable.
\end{tabular}
\end{description}

\noindent
{\bf Description}\\
This function will allocate and fill an array with handles to 
agents which currently reside in an agency. All agents will be listed
regardless of agent status. \\

\noindent
{\bf Example}\\
\noindent
% FIXME: Need an example here. 
%{\footnotesize\verbatiminput{../demos/multiple_agency_example/server.c}}

\noindent
{\bf See Also}\\
MC\_GetAgent().

%\CPlot::\DataThreeD(), \CPlot::\DataFile(), \CPlot::\Plotting(), \plotxy().\\
