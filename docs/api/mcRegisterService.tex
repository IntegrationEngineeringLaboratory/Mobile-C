\rhead{\bf mc\_RegisterService()}
\noindent
\vspace{5pt}
\rule{6.5in}{0.015in}
\noindent
\phantomsection
{\LARGE \bf mc\_RegisterService()\index{mc\_RegisterService()}}\\
\addcontentsline{toc}{section}{mc\_RegisterService()}
\label{api:mc_RegisterService()}

\noindent
{\bf Synopsis}\\
{\bf \#include $<$libmc.h$>$}\\
{\bf int mc\_RegisterService}({\bf MCAgent\_t} $agent$, {\bf int} agentID, {\bf const char} agentName, {\bf char**} serviceNames, {\bf int} numServices);\\

\noindent
{\bf Purpose}\\
Registers an agent service with an agency Directory Facilitator.\\

\noindent
{\bf Return Value}\\
The function returns 0 on success and non-zero otherwise.\\

\noindent
{\bf Parameters}
\vspace{-0.1in}
\begin{description}
\item
\begin{tabular}{p{23 mm}p{145 mm}} 
$agent$ & (Optional) An initialized mobile agent. \\
$agentID$ & (Optional) An agent id. \\
$agentName$ & (Optional) An agent name. \\
$serviceNames$ & A list of descriptive names for agent services. \\
$numServices$ & The number of services listed in the previous argument.
\end{tabular}
\end{description}

\noindent
{\bf Description}\\
This function is used to register agent services with an agency. Among
the optional arguments, either a valid agent must be supplied, or both
an agent ID and an agent name. Thus, services may be registered to an
agent which has not yet arrived at an agency by specifying the ID and name
of the agent.\\

\noindent
{\bf Example}\\
\noindent
{\footnotesize\verbatiminput{../demos/agent_space_functionality/mc_df_example/agent1.xml}}

\noindent
{\bf See Also}\\
MC\_RegisterService(), mc\_DeregisterService().

%\CPlot::\DataThreeD(), \CPlot::\DataFile(), \CPlot::\Plotting(), \plotxy().\\
