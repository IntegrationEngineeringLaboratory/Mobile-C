\rhead{\bf mc\_GetAgentName()}
\noindent
\vspace{5pt}
\rule{6.5in}{0.015in}
\noindent
\phantomsection
{\LARGE \bf mc\_GetAgentName()\index{mc\_GetAgentName()}}\\
\addcontentsline{toc}{section}{mc\_GetAgentName()}
\label{api:mc_GetAgentName()}

\noindent
{\bf Synopsis}\\
{\bf \#include $<$libmc.h$>$}\\
{\bf int mc\_GetAgentName}({\bf MCAgent\_t} $agent$);\\

\noindent
{\bf Purpose}\\
Get an agent's name. \\

\noindent
{\bf Return Value}\\
This function returns an agent's name. \\

\noindent
{\bf Parameters}
\vspace{-0.1in}
\begin{description}
\item
\begin{tabular}{p{10 mm}p{145 mm}} 
$agent$ & An initialized mobile agent.
\end{tabular}
\end{description}

\noindent
{\bf Description}\\
This function returns an agent's name. All agents have a self defined
descriptive name that may not be unique. This function gets the name of
an agent.\\

\noindent
{\bf Example}\\
\noindent
% FIXME: Example needed
%{\footnotesize\verbatiminput{../demos/mc_sample_app/korebot/service_provider_2.xml}}

\noindent
{\bf See Also}\\
mc\_GetAgentID().

%\CPlot::\DataThreeD(), \CPlot::\DataFile(), \CPlot::\Plotting(), \plotxy().\\
