\rhead{\bf MC\_AgentReturnDataGetSymbolAddr()}
\noindent
\vspace{5pt}
\rule{6.5in}{0.015in}
\noindent
\phantomsection
{\LARGE \bf MC\_AgentReturnDataGetSymbolAddr()\index{MC\_AgentReturnDataGetSymbolAddr()}}\\
\addcontentsline{toc}{section}{MC\_AgentReturnDataGetSymbolAddr()}
\label{api:MC_AgentReturnDataGetSymbolAddr()}

\noindent
{\bf Synopsis}\\
{\bf \#include $<$libmc.h$>$}\\
{\bf const void* MC\_AgentReturnDataGetSymbolAddr}({\bf MCAgent\_t} agent, {\bf int} task\_num);\\

\noindent
{\bf Purpose}\\
Get a pointer to the array contained within a return agent.\\

\noindent
{\bf Return Value}\\
Returns a valid pointer or NULL on failure.\\

\noindent
{\bf Parameters}
\begin{itemize}
\item \texttt{agent} : A return agent.
\item \texttt{task\_num} : This variable chooses which task within an agent to
retrieve the array.
\end{itemize}


\noindent
{\bf Description}\\
This function retrieves a pointer to the first element of an array returned by
a returning agent.

\noindent
{\bf Example}\\
\noindent
Please see the example for \texttt{MC\_AgentReturnArrayDim()} on page \pageref{api:MC_AgentReturnArrayDim()}.

\noindent
{\bf See Also}\\
\texttt{
  MC\_AgentReturnArrayDim(), MC\_AgentReturnArrayExtent(), MC\_AgentReturnArrayNum(),
  MC\_AgentReturnDataSize(), MC\_AgentReturnDataType(),
}

%\CPlot::\DataThreeD(), \CPlot::\DataFile(), \CPlot::\Plotting(), \plotxy().\\
