\rhead{\bf mc\_AclWaitRetrieve()}
\noindent
\vspace{5pt}
\rule{6.5in}{0.015in}
\noindent
\phantomsection
{\LARGE \bf mc\_AclWaitRetrieve()\index{mc\_AclNew()}}\\
\addcontentsline{toc}{section}{mc\_AclWaitRetrieve()}
\label{api:mc_AclWaitRetrieve()}

\noindent
{\bf Synopsis}\\
{\bf \#include $<$libmc.h$>$}\\
{\bf int mc\_AclWaitRetrieve}({\bf mcAgent\_t} agent);\\

\noindent
{\bf Purpose}\\
Wait until there is a message in an agent's mailbox and retrieve it.\\

\noindent
{\bf Return Value}\\
An ACL message on success, or NULL on failure.\\

\noindent
{\bf Parameters}
\vspace{-0.1in}
\begin{description}
\item
\begin{tabular}{p{10 mm}p{145 mm}} 
$agent$ & An initialized mobile agent.
\end{tabular}
\end{description}

\noindent
{\bf Description}\\
This function is used to wait for activity on an empty mailbox. If this
function is called on an empty mailbox, the function will block indefinitely
until a message is posted to the mailbox. Once a message is posted, the
function will unblock and return the new message. \\

\noindent
{\bf Example}\\
\noindent
{\footnotesize \verbatiminput{../demos/FIPA_compliant_ACL_messages/fipa_test/test1.xml}}

\noindent
{\bf See Also}\\
\texttt{
  mc\_AclNew(), mc\_AclPost(), mc\_AclReply(), mc\_AclSend(), 
    \linebreak mc\_AclWaitRetrieve()
}

%\CPlot::\DataThreeD(), \CPlot::\DataFile(), \CPlot::\Plotting(), \plotxy().\\
