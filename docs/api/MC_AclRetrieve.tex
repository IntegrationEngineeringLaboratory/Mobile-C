\rhead{\bf MC\_AclRetrieve()}
\noindent
\vspace{5pt}
\rule{6.5in}{0.015in}
\noindent
\phantomsection
{\LARGE \bf MC\_AclRetrieve()\index{MC\_AclNew()}}\\
\addcontentsline{toc}{section}{MC\_AclRetrieve()}
\label{api:MC_AclRetrieve()}

\noindent
{\bf Synopsis}\\
{\bf \#include $<$libmc.h$>$}\\
{\bf int MC\_AclRetrieve}({\bf MCAgent\_t} agent);\\

\noindent
{\bf Purpose}\\
Retrieve a message from an agent's mailbox.\\

\noindent
{\bf Return Value}\\
An ACL message on success, or NULL if no messages are in the 
mailbox.\\

\noindent
{\bf Parameters}
\vspace{-0.1in}
\begin{description}
\item
\begin{tabular}{p{10 mm}p{145 mm}} 
$agent$ & An initialized mobile agent.
\end{tabular}
\end{description}

\noindent
{\bf Description}\\
This function is used to retrieve a message from an agent's mailbox. The
message are retrieved in FIFO order. If there are no messages in the
mailbox, the function will return NULL.\\

\noindent
{\bf Example}\\
\noindent
{\footnotesize\verbatiminput{../demos/FIPA_compliant_ACL_messages/fipa_test/test1.xml}}

\noindent
{\bf See Also}\\
\texttt{
  MC\_AclNew(), MC\_AclPost(), MC\_AclReply(), MC\_AclSend(), \linebreak
    MC\_AclWaitRetrieve()
}

%\CPlot::\DataThreeD(), \CPlot::\DataFile(), \CPlot::\Plotting(), \plotxy().\\
