\rhead{\bf MC\_AgentReturnArrayExtent()}
\noindent
\vspace{5pt}
\rule{6.5in}{0.015in}
\noindent
\phantomsection
{\LARGE \bf MC\_AgentReturnArrayExtent()\index{MC\_AgentReturnArrayExtent()}}\\
\addcontentsline{toc}{section}{MC\_AgentReturnArrayExtent()}
\label{api:MC_AgentReturnArrayExtent()}

\noindent
{\bf Synopsis}\\
{\bf \#include $<$libmc.h$>$}\\
{\bf int MC\_AgentReturnArrayExtent}({\bf MCAgent\_t} agent, {\bf int} task\_num, {\bf int} index);\\

\noindent
{\bf Purpose}\\
Get the extent of a dimension of an array contained within a return agent.\\

\noindent
{\bf Return Value}\\
Returns the extent of the dimension of the array and index \texttt{index}, or
-1 on failure.\\

\noindent
{\bf Parameters}
\begin{itemize}
\item \texttt{agent} : A return agent.
\item \texttt{task\_num} : This variable chooses which task within an agent to
retrieve the array dimension.
\item \texttt{index} : The index of the array dimension to retrieve.
\end{itemize}


\noindent
{\bf Description}\\
This function is used to retrieve the extent of a single dimension of an array
held by a returning agent. The \texttt{index} argument must be smaller than the
dimension of the array.

\noindent
{\bf Example}\\
\noindent
Please see the example for \texttt{MC\_AgentReturnArrayDim()} on page \pageref{api:MC_AgentReturnArrayDim()}.

\noindent
{\bf See Also}\\
\texttt{
  MC\_AgentReturnArrayDim(), MC\_AgentReturnArrayNum()
}

%\CPlot::\DataThreeD(), \CPlot::\DataFile(), \CPlot::\Plotting(), \plotxy().\\
