\rhead{\bf MC\_TerminateAgent()}
\noindent
\vspace{5pt}
\rule{6.5in}{0.015in}
\noindent
{\LARGE \bf MC\_TerminateAgent()\index{MC\_TerminateAgent()}}\\
\phantomsection
\addcontentsline{toc}{section}{MC\_TerminateAgent()}

\noindent
{\bf Synopsis}\\
{\bf \#include $<$libmc.h$>$}\\
{\bf int MC\_TerminateAgent}({\bf MCAgent\_t} $agent$);\\

\noindent
{\bf Purpose}\\
Terminate the execution of a mobile agent in an agency.\\

\noindent
{\bf Return Value}\\
The function returns 0 on success and an error code on failure.\\

\noindent
{\bf Parameters}
\vspace{-0.1in}
\begin{description}
\item               
\begin{tabular}{p{10 mm}p{145 mm}}
$agent$ & A valid mobile agent. 
\end{tabular}
\end{description}

\noindent
{\bf Description}\\
This function halts a running mobile agent. 
The Ch interpreter is left intact. 
The mobile agent may still reside in the agency in MC\_AGENT\_NEUTRAL mode if 
the mobile agent is tagged as 'persistent', or is terminated and flushed 
otherwise.\\

\noindent
{\bf Example}\\
This function is identical to the agent-space counterpart. Please see the example
listed under mc\_TerminateAgent() on page \pageref{api:mc_TerminateAgent()}.\\
\noindent

\noindent
{\bf See Also}\\

%\CPlot::\DataThreeD(), \CPlot::\DataFile(), \CPlot::\Plotting(), \plotxy().\\
