\rhead{\bf mc\_CondBroadcast()}
\noindent
\vspace{5pt}
\rule{6.5in}{0.015in}
\noindent
\phantomsection
{\LARGE \bf mc\_CondBroadcast()\index{mc\_CondBroadcast()}}\\
\addcontentsline{toc}{section}{mc\_CondBroadcast()}

\noindent
{\bf Synopsis}\\
{\bf int mc\_CondBroadcast}({\bf int} $id$);\\

\noindent
{\bf Purpose}\\
Signal all mobile agents and threads which are waiting on a condition 
variable.\\

\noindent
{\bf Return Value}\\
This function returns 0 if the condition variable is successfully found and 
signalled.
It returns non-zero if the condition variable was not found.\\

\noindent
{\bf Parameters}
\vspace{-0.1in}
\begin{description}
\item
\begin{tabular}{p{10 mm}p{145 mm}}
$id$ & The id of the condition variable to signal.
\end{tabular}
\end{description}

\noindent
{\bf Description}\\
This function is used to signal all other mobile agents and threads that are 
waiting on a Mobile-C condition variable. 
The function that calls {\bf mc\_CondBroadcast()} must know beforehand the id of 
the condition variable which a mobile agent might be waiting on.\\

\noindent
{\bf Example}\\
Please see Program \vref{prog:mobileagent2_ex5.xml} and 
Program \vref{prog:binary_sync_example_agent} in Chapter 
\ref{chap:synchronization}.
\noindent
%Compare with output for examples in \CPlot::\Arrow(), \CPlot::\AutoScale(),
%\CPlot::\DisplayTime(), \CPlot::\Label(), \CPlot::\TicsLabel(), 
%\CPlot::\Margins(), \CPlot::\BoundingBoxOffsets(), \CPlot::\TicsDirection(),\linebreak
%\CPlot::\TicsFormat(), and \CPlot::\Title().
%{\footnotesize\verbatiminput{template/example/Data2D.ch}}

\noindent
{\bf See Also}\\
mc\_CondDelete(), mc\_CondInit(), mc\_CondSignal().\\

%\CPlot::\DataThreeD(), \CPlot::\DataFile(), \CPlot::\Plotting(), \plotxy().\\
