\rhead{\bf MC\_FindAgentByID()}
\noindent
\vspace{5pt}
\rule{6.5in}{0.015in}
\noindent
{\LARGE \bf MC\_FindAgentByID()\index{MC\_FindAgentByID()}}\\
\phantomsection
\addcontentsline{toc}{section}{MC\_FindAgentByID()}

\noindent
{\bf Synopsis}\\
{\bf \#include $<$libmc.h$>$}\\
{\bf MCAgent\_t MC\_FindAgentByID}({\bf MCAgency\_t} $agency$, {\bf int} $id$);\\

\noindent
{\bf Purpose}\\
Find a mobile agent by its ID number in a given agency.\\

\noindent
{\bf Return Value}\\
The function returns an {\bf MCAgent\_t} object on success or NULL on failure.\\

\noindent
{\bf Parameters}
\vspace{-0.1in}
\begin{description}
\item               
\begin{tabular}{p{10 mm}p{145 mm}}
$agency$ & An agency handle.\\
$id$ & An integer representing a mobile agent's ID number.
\end{tabular}
\end{description}

\noindent
{\bf Description}\\
This function is used to find and retrieve a pointer to an existing running 
mobile agent in an agency by the mobile agent's ID number.\\

\noindent
{\bf Example}\\
This function is equivalent to the agent-space version. Please see the
example for mc\_FindAgentByID() listed on page \vref{api:mc_FindAgentByID()}.
\noindent
%Compare with output for examples in \CPlot::\Arrow(), \CPlot::\AutoScale(),
%\CPlot::\DisplayTime(), \CPlot::\Label(), \CPlot::\TicsLabel(), 
%\CPlot::\Margins(), \CPlot::\BoundingBoxOffsets(), \CPlot::\TicsDirection(),\linebreak
%\CPlot::\TicsFormat(), and \CPlot::\Title().
%{\footnotesize\verbatiminput{template/example/Data2D.ch}}

\noindent
{\bf See Also}\\
MC\_FindAgentByName()

%\CPlot::\DataThreeD(), \CPlot::\DataFile(), \CPlot::\Plotting(), \plotxy().\\
