\rhead{\bf MC\_AddAgent()}
\noindent
\vspace{5pt}
\rule{6.5in}{0.015in}
\noindent
\phantomsection
{\LARGE \bf MC\_AddAgent()\index{MC\_AddAgent()}}\\
\addcontentsline{toc}{section}{MC\_AddAgent()}
\label{api:MC_AddAgent()}

\noindent
{\bf Synopsis}\\
{\bf \#include $<$libmc.h$>$}\\
{\bf int MC\_AddAgent}({\bf MCAgency\_t} $agency$, {\bf MCAgent\_t} $agent$);\\

\noindent
{\bf Purpose}\\
Add a mobile agent into an agency.\\

\noindent
{\bf Return Value}\\
The function returns 0 on success and non-zero otherwise.\\

\noindent
{\bf Parameters}
\vspace{-0.1in}
\begin{description}
\item
\begin{tabular}{p{10 mm}p{145 mm}} 
$agency$ & An initialized agency handle to add an agent to.\\
$agent$ & An initialized mobile agent.
\end{tabular}
\end{description}

\noindent
{\bf Description}\\
This function adds a mobile agent to an already running agency.

Please note, please lock the agent queue with function
\texttt{MC\_QueueWRLock(agency, MC\_QUEUE\_AGENT);} prior
to calling the \texttt{MC\_AddAgent()} function.

\noindent
{\bf Example}\\
\noindent
{\footnotesize\verbatiminput{../demos/miscellaneous/multiple_agency_example/server.c}}

\noindent
{\bf See Also}\\

%\CPlot::\DataThreeD(), \CPlot::\DataFile(), \CPlot::\Plotting(), \plotxy().\\
