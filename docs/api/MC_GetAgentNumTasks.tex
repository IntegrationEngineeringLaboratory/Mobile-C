\rhead{\bf MC\_GetAgentNumTasks()}
\noindent
\vspace{5pt}
\rule{6.5in}{0.015in}
\noindent
{\LARGE \bf MC\_GetAgentNumTasks()\index{MC\_GetAgentNumTasks()}}\\
\phantomsection
\addcontentsline{toc}{section}{MC\_GetAgentNumTasks()}

\noindent
{\bf Synopsis}\\
{\bf \#include $<$libmc.h$>$}\\
{\bf int MC\_GetAgentNumTasks}({\bf MCAgent\_t} $agent$);\\

\noindent
{\bf Purpose}\\
Return the total number of tasks a mobile agent has. \\

\noindent
{\bf Return Value}\\
This function returns a non negative integer on success and a negative 
integer on failure.\\

\noindent
{\bf Parameters}
\vspace{-0.1in}
\begin{description}
\item               
\begin{tabular}{p{10 mm}p{145 mm}}
$agent$ & A MobileC agent. 
\end{tabular}
\end{description}

\noindent
{\bf Description}\\
This function returns the total number of tasks that an agent has. 
It counts all tasks: those that have been completed, those that are in 
progress, and those that have not yet started.\\

\noindent
{\bf Example}
\begin{verbatim}
int i;
MCAgent_t agent;

/* More code here */

i = MC_GetAgentNumTasks(agent);
printf("The agent has %d tasks.\n", i);
\end{verbatim}

\noindent
The previous piece of code retrieves the nuber of tasks that an agent has and
prints it to standard output.\\
%Compare with output for examples in \CPlot::\Arrow(), \CPlot::\AutoScale(),
%\CPlot::\DisplayTime(), \CPlot::\Label(), \CPlot::\TicsLabel(), 
%\CPlot::\Margins(), \CPlot::\BoundingBoxOffsets(), \CPlot::\TicsDirection(),\linebreak
%\CPlot::\TicsFormat(), and \CPlot::\Title().
%{\footnotesize\verbatiminput{template/example/Data2D.ch}}

\noindent
{\bf See Also}\\

%\CPlot::\DataThreeD(), \CPlot::\DataFile(), \CPlot::\Plotting(), \plotxy().\\
