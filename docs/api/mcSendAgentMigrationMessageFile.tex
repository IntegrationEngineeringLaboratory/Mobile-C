\rhead{\bf mc\_SendAgentMigrationMessageFile()}
\noindent
\vspace{5pt}
\rule{6.5in}{.01in}
\noindent
{\LARGE \bf mc\_SendAgentMigrationMessageFile()\index{mc\_SendAgentMigrationMessageFile()}}\\
\phantomsection
\addcontentsline{toc}{section}{mc\_SendAgentMigrationMessageFile()}

\noindent
{\bf Synopsis}\\
%{\bf \#include $<$mobilec.h$>$}\\
{\bf int mc\_SendAgentMigrationMessageFile}({\bf const char *}$filename$, {\bf const char *}$hostname$, {\bf int} $port$);\\

\noindent
{\bf Purpose}\\
Send an ACL mobile agent message saved as a file to a remote agency.\\

\noindent
{\bf Return Value}\\
The function returns 0 on success and non-zero otherwise.\\

\noindent
{\bf Parameters}
\vspace{-0.1in}
\begin{description}
\item
\begin{tabular}{p{20 mm}p{135 mm}}
$filename$ & The ACL mobile agent message file to be sent.\\
$hostname$ & The hostname of the remote agency. It can be in number-dot 
format or hostname format, i.e., 169.237.104.199 or machine.ucdavis.edu.\\
$port$ & The port number on which the remote agency is listening. 
\end{tabular}
\end{description}

\noindent
{\bf Description}\\
This function is used to send an XML based ACL mobile agent message, which 
is saved as a file, to a remote agency.\\ 

\noindent
{\bf Example}\\
Please see the example for MC\_SendAgentMigrationMessageFile() on page
\pageref{api:MC_SendAgentMigrationMessageFile()}.
\noindent
%Compare with output for examples in \CPlot::\Arrow(), \CPlot::\AutoScale(),
%\CPlot::\DisplayTime(), \CPlot::\Label(), \CPlot::\TicsLabel(), 
%\CPlot::\Margins(), \CPlot::\BoundingBoxOffsets(), \CPlot::\TicsDirection(),\linebreak
%\CPlot::\TicsFormat(), and \CPlot::\Title().
%{\footnotesize\verbatiminput{template/example/Data2D.ch}}

\noindent
{\bf See Also}\\

%\CPlot::\DataThreeD(), \CPlot::\DataFile(), \CPlot::\Plotting(), \plotxy().\\
