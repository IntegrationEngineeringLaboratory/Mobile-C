\rhead{\bf MC\_SendAgentFile()} 
\noindent
\vspace{5pt}
\rule{6.5in}{.01in}
\noindent
{\LARGE \bf MC\_SendAgentFile()\index{MC\_SendAgentFile()}}\\ 
\phantomsection
\addcontentsline{toc}{section}{MC\_SendAgentFile()}
\label{api:MC_SendAgentFile()}

\noindent
{\bf Synopsis}\\
{\bf \#include $<$libmc.h$>$}\\
{\bf int MC\_SendAgentFile}({\bf MCAgency\_t} $agency$, {\bf char *}$filename$);\\

\noindent
{\bf Purpose}\\
Send an ACL mobile agent message saved as a file to a remote agency.

\noindent
{\bf Return Value}\\
The function returns 0 on success and non-zero otherwise.\\

\noindent
{\bf Parameters}
\vspace{-0.1in}
\begin{description}
\item
\begin{tabular}{p{20 mm}p{135 mm}}
$agency$ & A handle associated with an agency from which to send the ACL 
mobile agent message. A NULL pointer can be used to send the ACL message 
from an unspecified agency.\\ 
$filename$ & The ACL mobile agent message file to be sent.\\
\end{tabular}
\end{description}

\noindent
{\bf Description}\\
This function is used to send an XML based ACL mobile agent message, which 
is saved as a file, to a remote agency. \\

\noindent
{\bf Example}\\
\noindent
{\footnotesize\verbatiminput{../demos/getting_started/hello_world/client.c}}

\noindent
{\bf See Also}\\

%\CPlot::\DataThreeD(), \CPlot::\DataFile(), \CPlot::\Plotting(), \plotxy().\\
