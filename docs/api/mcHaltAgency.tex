\rhead{\bf mc\_HaltAgency()}
\noindent
\vspace{5pt}
\rule{6.5in}{0.015in}
\noindent
\phantomsection
{\LARGE \bf mc\_HaltAgency()\index{mc\_HaltAgency()}}\\
\addcontentsline{toc}{section}{mc\_HaltAgency()}
\label{api:mc_HaltAgency()}

\noindent
{\bf Synopsis}\\
{\bf \#include $<$libmc.h$>$}\\
{\bf int mc\_HaltAgency}({void});\\

\noindent
{\bf Purpose}\\
This function halts the execution of an agency. \\

\noindent
{\bf Return Value}\\
The function returns 0 on success and non-zero otherwise.\\

\noindent
{\bf Parameters}
None.\\

\noindent
{\bf Description}\\
This function halts the primary threads of an agency, such as the ACC, AMS,
message handlers, etc. If any thread is busy with
a particular task, it will halt as soon as the task is finished. Note that
this function does not halt the execution of any agents which may be 
performing tasks. Agents performing tasks may not rely on the primary 
Mobile-C threads, such as the ACC, AMS, etc., and thus may not halt upon
calling this function. \\

\noindent
{\bf Example}\\
\noindent
% FIXME: Need an example here
%{\footnotesize\verbatiminput{../demos/multiple_agency_example/server.c}}

\noindent
{\bf See Also}\\
mc\_ResumeAgency().

%\CPlot::\DataThreeD(), \CPlot::\DataFile(), \CPlot::\Plotting(), \plotxy().\\
