\rhead{\bf MC\_AgentReturnArrayNum()}
\noindent
\vspace{5pt}
\rule{6.5in}{0.015in}
\noindent
\phantomsection
{\LARGE \bf MC\_AgentReturnArrayNum()\index{MC\_AgentReturnArrayNum()}}\\
\addcontentsline{toc}{section}{MC\_AgentReturnArrayNum()}
\label{api:MC_AgentReturnArrayNum()}

\noindent
{\bf Synopsis}\\
{\bf \#include $<$libmc.h$>$}\\
{\bf int MC\_AgentReturnArrayNum}({\bf MCAgent\_t} agent, {\bf int} task\_num);\\

\noindent
{\bf Purpose}\\
Get the total number of elements in an array returned by a returning agent.\\

\noindent
{\bf Return Value}\\
Returns the total number of elements in the array, or
-1 on failure.\\

\noindent
{\bf Parameters}
\begin{itemize}
\item \texttt{agent} : A return agent.
\item \texttt{task\_num} : This variable chooses which task within an agent to
retrieve the number of array elements.
\end{itemize}


\noindent
{\bf Description}\\
This function is used to find the total number of elements of a returned 
array. For example, if a 3 by 4 two-dimensional array is returned, this
function will report that there are 12 elements in the array.

\noindent
{\bf Example}\\
\noindent
Please see the example for \texttt{MC\_AgentReturnArrayDim()} on page \pageref{api:MC_AgentReturnArrayDim()}.

\noindent
{\bf See Also}\\
\texttt{
  MC\_AgentReturnArrayDim(), MC\_AgentReturnArrayExtent()
}

%\CPlot::\DataThreeD(), \CPlot::\DataFile(), \CPlot::\Plotting(), \plotxy().\\
