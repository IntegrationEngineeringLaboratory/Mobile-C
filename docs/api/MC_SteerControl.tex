\rhead{\bf MC\_SteerControl()}
\noindent
\vspace{5pt}
\rule{6.5in}{0.015in}
\noindent
\phantomsection
{\LARGE \bf MC\_SteerControl()\index{MC\_SteerControl()}}\\
\addcontentsline{toc}{section}{MC\_SteerControl()}

\noindent
{\bf Synopsis}\\
{\bf \#include $<$libmc.h$>$}\\
{\bf int MC\_SteerControl}({\bf void});\\

\noindent
{\bf Purpose}\\
This function is used to enable Mobile-C as a steerable computational
platform. See the example following for more information, as well
as the demo provided in the directory demos/steer\_example.\\

\noindent
{\bf Return Value}\\
This function returns the current steer command. The command is of type
{\bf enum MC\_Steer\_Command\_e}. This enumerated type contains the following 
definitions:\\
\begin{tabular}{p{55 mm}p{120 mm}}
MC\_RUN & Continue the algorithm. \\
MC\_SUSPEND & Pause the algorithm. \\
MC\_RESTART & Restart the algorithm from the beginning. \\
MC\_STOP & Stop the algorithm.
\end{tabular}\\

\noindent
{\bf Description}\\
{\bf MC\_SteerControl} controls the execution of an algorithm in binary space. 
This function is meant to retrieve the current requested command for the algorithm,
but it is up to the algorithm implementation to actually implement these
behaviours. See the example and the demo for more details.\\ \noindent
{\bf Example}\\
\noindent
{\footnotesize \verbatiminput{../demos/miscellaneous/steer_example/server.c}}
%Compare with output for examples in \CPlot::\Arrow(), \CPlot::\AutoScale(),
%\CPlot::\DisplayTime(), \CPlot::\Label(), \CPlot::\TicsLabel(), 
%\CPlot::\Margins(), \CPlot::\BoundingBoxOffsets(), \CPlot::\TicsDirection(),\linebreak
%\CPlot::\TicsFormat(), and \CPlot::\Title().
%{\footnotesize\verbatiminput{template/example/Data2D.ch}}

\noindent
{\bf See Also}\\
MC\_Steer()

%\CPlot::\DataThreeD(), \CPlot::\DataFile(), \CPlot::\Plotting(), \plotxy().\\
